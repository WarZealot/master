%Vorwort
%\topskip0pt
%\vspace*{\fill}
\null\vfill
\begin{center}
	\subsubsection*{Zusammenfassung}
\end{center}
Das sich gerade stark weiterentwickelnde Gebiet des Internets der Dinge ist derzeit noch sehr fragmentiert. Zwei bis dato völlig unabhängige Bereiche von IoT sind Smart Home und webbasierte Task Automation Services. Dennoch weisen sie viele Gemeinsamkeiten auf und ein Verschmelzen der beiden Ansätze verspricht Vorteile und Lösungen in Aspekten, wie Reaktionszeiten, Durchsatz und Sicherheit.

Im Rahmen dieser Arbeit gilt es diese Annahmen in der Praxis zu prüfen.
Zunächst sollen die Gemeinsamkeiten und Unterschiede der Ansätze im Detail betrachtet werden, wonach konkrete Anforderungen an einen Prototypen zu formulieren sind. 
Daraufhin wird ein kurzer Einblick in den Eclipse Open IoT Stack gegeben, auf dessen Basis der Prototyp entworfen und implementiert wird. 
Anschließend wird er einem umfangreichen Test unterzogen, wodurch die zugrundeliegenden Annahmen evaluiert werden sollen.


 %Hierfür soll ein Prototyp, basierend auf dem Eclipse Open IoT Stack, entworfen und umgesetzt werden. 

%\vspace*{\fill}
\vfill\null