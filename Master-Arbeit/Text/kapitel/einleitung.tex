% einleitung.tex
\chapter{Einleitung}
\section{Motivation und Hintergrund}
Motivation und Hintergrund

\section{Ziele der Arbeit}
Ziel dieser Arbeit ist es die Welten von Smart Home und webbasierten Task Automation Services zusammen zu bringen. Es soll ein Demonstrator entwickelt werden, der die  Funktionalitäten beider Ansätze kombiniert und in einem gemeinsamen Kontext anbietet. Konkreter sind die Ziele in Sektion \ref{sec:ziele} erläutert.


\section{Aufbau der Arbeit}
Nach der Einleitung wird der aktuelle Stand der Forschung bezüglich \textit{Internet der Dinge}, \textit{SmartHome} und \textit{Task Automation Services} vorgestellt und die Ziele der Arbeit daraus abgeleitet. Daraufhin wird in Kapitel 3 das Framework Eclipse SmartHome (ESH) vorgestellt. In Kapitel 4 wird konkret betrachtet, wie die Funktionalitäten eines Task Automation Services in ESH integriert werden können.

Im folgenden Kapitel 5 wird die Implementierung des Entwurfs mit Hilfe von entsprechenden
Diagrammen vorgestellt. Der Quell-Code ist auf der mit der Arbeit mitgelieferten
CD einsehbar. Danach wird in Abschnitt 6 die Implementierung evaluiert. Schließlich folgt
noch eine kurze Zusammenfassung der geleisteten Arbeit und ein Ausblick auf mögliche
Weiterentwicklungen in Kapitel 7.

